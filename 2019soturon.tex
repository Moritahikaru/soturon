\documentclass[12pt]{jarticle}
\usepackage[a4paper,text={155mm,230mm},centering]{geometry}
\usepackage{amssymb,amsmath,amsthm}
\usepackage[dvips]{graphicx}
\usepackage{fancyhdr}
\pagestyle{fancy}
\lhead{ワイヤレス給電の最適な周波数を実験による決定方法について}
\chead{}
\rhead{\thepage}
\lfoot{}
\cfoot{}
\rfoot{森田光流}
\usepackage{fancybox}

\begin{document}
	
%%%%%%%%%%%%%%%%%%%%%%%%%%%%%%%%%表紙
\thispagestyle{empty}

\vspace*{20mm}
\begin{center}
	{\Large 令和元年度 環境ロボティクス学科 卒業論文}
\end{center}
\vspace{10mm}
\begin{center}
	{\Huge ワイヤレス給電の最適な周波数を実験による決定方法について}
\end{center}
\vspace{90mm}
\begin{center}
	{\Large 令和2年 (2020) 2月14日}
\end{center}
\begin{center}
	{\Large 森田 光流}
\end{center}
\begin{center}
	{\Large 指導教員:穂高一条教授}
\end{center}

\clearpage
%%%%%%%%%%%%%%%%%%%%%%%%目次

\tableofcontents

\clearpage
%%%%%%%%%%%%%%%%%%%%%%%本文

\section{緒言}
\subsection{背景}
 
\subsection{研究目的}
\clearpage
\section{謝辞}
 本研究の進行や本論文等の執筆にあたり,ご指導いただいた穂高一条教授に感謝の意を示すとともに深く御礼申し上げます.
また本研究を進めるにあたり多大なご指導・助言してくださった自動制御研究室の先輩方並びに,共に研究した同期のメンバー
にも感謝の意を示すとともに深く御礼申し上げます.最後になりましたが,お世話になりました宮崎大学工学部環境ロボティクス
学科の先生方,並びに大学関係各位の皆様に心より感謝し,ここに御礼申し上げます.

\begin{thebibliography}{50}
	
\end{thebibliography}
\clearpage
\section{付録}

\end{document}

